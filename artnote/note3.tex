%%%%%%%%%%%%%%%%%%%%%%%%%%%%%%%%%%%%%%%%%%%%%%%%%%%%%%%%%%%%%%%页面与标题式样

\usepackage{geometry}
% 利用 geometry 可以很方便的设置页面的大小。由于可以自动居中排放页面,自动计算并平衡页面各部分如页眉、页脚、左右边空等的大小,因此只需给出很少的信息就能得到满意的页面。
 
% rmpage 
% 提供了简单的命令来设置页面的大小,并通过调整页面的宽度确保文本在打印区域内。若你地页面需要特定地页面布局参数,最好还是使用上面的 geometry 宏包。
 
% layout 
% 显示文档的页面上各部分的设置。可用命令 layout 来得到本文档的页面设置的视图。是 LaTeX 标准的工具包 tools 之一。一般的 TeX 软件均包括此宏包。
 
% layouts
% 比 layout 功能更强大,可显示文档的页面上各部分的设置。包括文本在一页中的位置,图表等浮动对象的位置移动,以及章节标题的设计及其在目录中的形式等。
 
% multicol
% 提供了一新的环境,使得可在一页上使用单栏和多栏版式。是 LaTeX 标准的工具包 tools 之一。一般的 TeX 软件均包括此宏包。
 
\usepackage{fancyhdr}
% 用 fancyhdr 来设置页眉和页脚十分方便,而且可以在配合 CCT、CJK来设置中文的页眉等。

% rplain 
% 重新定义了 plain 页面式样,将页码放置在页面的左下角。在双面式样中,则分别为奇数页的左下角和偶数页的右下角。
 
% pageno 
% 可以将很方便的设置页码在页面上的放置位置。
 
% titling 
% 提供了一些命令用来控制由 maketitle 命令生成的文档标题的式样。
 
% titlesec
% 厌烦千篇一律的章节标题?那就试试 titlesec 吧!让你选择自己喜爱的标题式样,只需要几行简单的命令就足够了。
 
% sectsty
% 如同 titlesec 一样,提供了许多命令来使用户很方便地设计自己喜爱的章节标题的风格。
 
% fncychap
% 另一个设计标题式样的宏包,主要是针对章的标题。
 
% anysize
% 设定页面的大小,调整正文区和边空的大小。
 
% crop
% 提供不同形式的截角标记,并提供选项来使排版的内容居中,标记垂直和水平的中轴线等。
 
% fix2col 
% 修补了标准的 LaTeX 双栏版式的一些不尽如人意的地方。
 
% ragged2e 
% 提供了一些新的命令和环境来协助 LaTeX 断词,从而尽可能地使排版得到的输出比较整齐。
 
% scale
% 将整个的文档放大 1.44 (magstep2) 倍。
 
%%%%%%%%%%%%%%%%%%%%%%%%%%%%%%%%%%%%%%%%%%%%%%%%%%%%%%%%%%%%%浮动对象及标题设计 

%floatflt
%floatflt 宏包提供了 floatingfigure 和 floatingtable 两个环境,可将浮动图形或表格放置于文字段落的旁边。
 
%float
%利用该宏包可以定义自己喜欢的浮动对象的式样而不必拘泥于 LaTeX 所预定的设置。
 
%rotating
%可以将文本、表格、图形旋转,并提供了 sidewayfigure 和 sidewaystable 环境来使图形或表格横排。另外,也可以用 rotcaption 命令来只对图形或表格的标题加以横排。参见使用该宏包的例子(PDF)及其源码(LaTeX)。
 
%rotfloat 
%将 rotating 宏包和 float 宏包结合起来,通过对 float 宏包所定义的命令加以扩展,可以很方便的定义新的被旋转 90°或 270°的浮动对象。
 
%endfloat 
%将所有的浮动对象放置于文章的最后分类排出。如将浮动图形都放置于文章的最后名为 Figures  的一章中,浮动表格等也类似的排放。
 
%afterpage 
%提供命令 afterpage,该命令使得所有作为其参数给出的 LaTeX 命令在当前页结束后才被执行。
 
%placeins
%提供 FloatBarrier 命令,常用来解决过多未处理浮动图形的问题。
 
\usepackage{caption}
% 提供了多种命令来更方便的设计浮动图形和表格的<标题式样>。
 
%caption2 
%另一个功能强大的设计浮动对象的标题式样的宏包。参见该宏包的说明文档(英文 PDF )。
 
%sidecap
%轻松的得到标题在一边的浮动图形或表格。参见该宏包的说明文档(英文 PDF )。
 
%fltpage 
%如果遇上图形或表格太大,以至无法和标题放置于同一页的情况该怎么办?那就试试 fltpage 吧。
 
\usepackage{subfigure} 
%可以将一组图形或表格放在一个 figure 或 table 环境中,而每幅图形或表格都保持一定的独立性,可以有自己的标题等。例如你想把几幅图形分别编号为: Figure 1(a), 1(b), 1(c),..., 就可以用此宏包的 subfigure 命令来实现。另外,还提供 subtable 命令来处理表格的情况。
 
%%%%%%%%%%%%%%%%%%%%%%%%%%%%%%%%%%%%%%%%%%%%%%%%%%%%%%%%%%%%%%%%%%%生成与插入图形 
%LaTeX2e Graphics 宏包套件
%LaTeX 中插图所必备,是 LaTeX2e 所带的标准宏包。对不同的 DVI 驱动, 提供了对 EPS, PS, PDF, TIFF, JPEG 等图形格式的支持。另外,该宏包还通过 color 宏包提供了对色彩的支持。有关使用方法和例子可参见本站的 LaTeX2e 插图指南。
 
%MetaPost 
%基于 MetaFont 的绘图语言。它的一些语法、命令等都和 MetaFont 类似,但不同的是它的输出为 PostScript 而不是位图。MetaPost 的绘图指令可以很好地融合在 TeX/LaTeX 文件中,在运行 tex 或 latex 进行编译的过程中生成 PostScript 图形并插入到文档中。特别需要指出的事,尽管 pdftex/pdflatex 不支持 EPS, PS 格式的图形,但 MetaPost 的输出却可以很容易地在其中被使用。详见 MetaPost 简介、用户指南(英文 PDF )。
 
%PStricks
%功能强大的绘图宏包,支持在 TeX/LaTeX 文件中直接使用 PostScript 命令,可以让你在文档中轻易而举地得到各种 PostScript 的图形、文字效果。使用该宏包地文档需要用 dvips 等转换为PS 文件后才能预览。另外,该宏包不能和 pdftex/pdflatex 配合使用。详见 PStricks 的用户手册 part 1, part2, part3, part4 (英文 PDF )。
 
%XYpic 
%为在 TeX/LaTeX 文件中绘制 graph 和 diagrams 提供强大的支持。它可以和 Plain TeX, AMSTeX, LaTeX, 甚至 PDFTeX 一起配合使用。使用 XYpic 可以很方便的得到各种曲线,箭头,多边形,直方图等等。更详细的内容可参考 XYpic 用户指南和 XYpic 参考手册(英文 PDF)。
 
%psfrag 
%允许用 LaTeX 的文本和公式来替代 EPS 图形文件中的字符。在 CJK, CCT 等中文环境下,可以使用 psfrag 将图形中的标记字符替换所需的中文文本。
 
%pspicture 
%使用 PostScript special 重新实现了 LaTeX 的 picture 环境,使得可以设定任意角度和粗细的线段,对圆的大小也没有了限制。
 
%texdraw 
%提供了许多命令来绘制各种式样的线段,bezier 曲线、圆、箭头等。也可以用不同的灰度来填充区域,在所绘制的图形上放置文本、数学符号。需要 PostScript 的支持。
 
\usepackage{picins} 
%picins 宏包定义了一个命令 parpic 命令,允许将图形等 LaTeX 对象放置在文本段落中。并且,设定适当的参数,可把该对象置于一带框的盒子,有阴影的盒子等等。
 
\usepackage{picinpar} 
%picinpar 宏包定义了一个基本的环境 window,还有两个变体 figwindow 和 tabwindow。允许在文本段落中打开一个“窗口”, 在其中放入图形、文字和表格等。。
 
%wrapfig 
%wrapfig 宏包提供了一个 wrapfigure 环境来排版窄小的图形,使得该图形位于文本的一边,并使文本在其边上折行。
 
eso-pic 
可以很容易地在文档的每一页上都加上一幅或几幅图形。比较适合于用来得到水印效果。
 
%overpic 
%允许直接将 LaTeX 对象放置到 一幅图形上,而不是通过对图形上已有的标记进行替换来实现。overpic 宏包中定义了一个 overpic 环境,它有效地将 picture 环境和 includegraphics 命令结合起来。 使得 picture 环境的维数和插入的 EPS 图形的维数相同。 这样就可以很容易地把 LaTeX 的命令放到图形上的任何指定位置。同时,还可以在图形上加上标尺以方便定位。参见其所附的两个示例:一(使用绝对位置),二(使用相对位置)。
 
%epic 和 eepic 
%epic 提供了对 LaTeX picture 环境的有限的扩展。而 eepic 宏包则是在 epic 的基础上更进一步扩展了 LaTeX 的 picture 环境,使得可以画出任意角度的线段,任意大小的圆,更多的线段粗细的选择等等。
 
%trees 
%很容易地画出任意大小的树形图。
 
%curves 
%不需要太多的 TeX memory, 就能得到各种具有连续角度的曲线,包括 bezrer 曲线,虚线等。
 
%%%%%%%%%%%%%%%%%%%%%%%%%%%%%%%%%%%%%%%%%%%%%%%%%%%%%%%%%%%%%%%%%%%%%%%%%表格与列表 
\usepackage{array} 
%增强了 tabular 环境的功能,可以更好的排版表格。
 
\usepackage{longtable}
%如果表格太长,超过了一页时,就可以试试 longtable 宏包所定义的 longtable 环境。
 
supertabular 
自动计算表格的高度,把超出页面的表格部分放置在下一页。
 
tabularx 
提供了新的表格环境 tabular*、tabularx,可以设定表格的宽度。
 
ltxtable 
简单说,就是 longtable 和 tabularx 两个宏包的结合。
 
colortbl 
利用该宏包可以设置表格中行、列等前景和背景色,从而得到彩色表格。
 
dcolumn 
让你感觉到在表格中将小数点对齐不再是一件麻烦的事情。
 
multirow
如果表各种某一单元横跨两行以上,就要用 multirow 了。
 
hhline
在表格中用 hhline 得到的结果就如同 hline 或 hlinehline,当然在和垂直线的交叉处会有所不同。
 
slashbox
可在表格的单元格中画上一斜线。
 
booktabs
让你的表格中使用不同粗细的横线来划分行。
 
mdwtab
重新实现了标准的 LaTeX2e 的 array 和 tabular 的功能,并增加了新的内容。。
 
paralist
提供新的列表环境,可以将 itemize 和 enumerate 列表排放在一段落中。
 
shortlst
专门用来排版列表项都很短的 itemize 和 enumerate 环境。
 
enumerate
给 enumerate 环境增加了一可选项,用来设定列表项的数字的形式。
 
multienum
支持将 enumerate 环境中的列表项用多列排出,即在一行中可以排出多个列表项。同时,提供了命令来设置每行中列表项的个数。
 
%%%%%%%%%%%%%%%%%%%%%%%%%%%%%%%%%%%%%%%%%%%%%%%%%%%%%%%%%%%%%%%%%%%%%%%%目录与索引
tocloft
提供了让你自己控制目录的式样的手段。
 
titletoc
设计自己喜欢的目录排版形式。
 
multitoc
允许在文档中只将目录,包括图形和表格目录用两栏或多栏排版。
 
minitoc
使用该宏包可以将每一章的目录放置在该章的任何地方(一般在开始或结尾部分)。
 
tocbibind
使用该宏包可以将参考文献或索引等放置到目录中去。
 
shorttoc
使用该宏包可以在正式的目录前生成一个比较简略的目录,可以方便读者了解文档内容。这在排版比较大的书籍时很有用。
 
tocvsec2
该宏包可以控制出现目录里每一章中编号的级别和/或是否给其编号。
 
makeindex
makeindex 不仅是一个 LaTeX 宏包,还有一个专门的同名应用程序来帮助生成 LaTeX 文档的索引。
 
nomencl
利用 makeindex 快速创建自己的符号命名列表。
 
%%%%%%%%%%%%%%%%%%%%%%%%%%%%%%%%%%%%%%%%%%%%%%%%%%%%%%%%%%%%%%%%%%%%%%%参考文献
bibtex
作为 LaTeX 的一个辅助程序,BibTeX 通过搜索一个或多个数据库,自动为 LaTeX 文档构造参考文献。
 
natbib
重新实现了 LaTeX 的 cite 命令,使得既可以使用“作者——年代”形式的文献索引,也可使用通常的数字编号形式的文献索引。。
 
footbib
定义了 footcite 命令,使得由该命令得到的参考文献的引用像脚注一样被放置在页面的底部 。
 
custom-bib/makebst
利用标准的参考文献式样文件,设计自己的可供 bibtex 使用的式样文件。
 
tocbibind
使用该宏包可以将参考文献或索引等放置到目录中去。
 
bibentry
使用该宏包可以在文本的任何地方放置参考文献的条目。
 
bibunits
使用该宏包允许文档的不同部分有各自的参考文献。这些部分可以是章、节或 bibunit 环境。
 
listbib
该宏包可以用来排版 BibTeX 的数据库文件,而且使用很少的 TeX 存储空间。这就使得可以排版很大的参考文献数据库文件。
 
gloss
使用该宏包可以借助于 bibtex 创建文档尾部的注释表(glossary)。
 
mcite
使用该宏包可以在文中同时对多个参考文献的关键词进行引用。
 
varioref
该宏包定义了多个交叉引用命令,这些命令都是 LaTeX 的 ref 命令加上一些文本后得到的。而这些加上的文本可以很方便地被替换为不同的语言,例如中文等。
 
fancyref
该宏包的引用命令 Fref 可以在根据你的标记的前缀给出不同的引用文字。比如 Fref{eq:first} 会给出“Equation (1) on page 2”,而 Fref{sec:first} 则会给出“Section 1 on page 2”。当然,这些前缀和文本的形式你都可以自己来设定。
 
prettyref
该宏包为 LaTeX 的交叉引用机制提供了附加功能,使得使用者可以预先设置所有类型的标记(label),和 fancyref 的功能差不多。
 
%%%%%%%%%%%%%%%%%%%%%%%%%%%%%%%%%%%%%%%%%%%%%%%%%%%%%%%%%%%%%%%%%%%%%%数学与化学公式
AMSLaTeX
作为 AMSTeX 在 LaTeX 中地实现,AMSLaTeX 包括两部分,一是 amsmath 宏包,主要的目的是用来排版数学符号和公式,其中专门有 amsthm 宏包,提供对定理的排版。另一部分是 amscls,提供了美国数学会要求的论文和书籍的格式。
 
AMS Fonts
美国数学会还提供一套的数学符号的字库,这套字库中增加了很多 TeX 的标准字库 Computer Modern 所没有的一些数学符号,如粗体数学符号等。
 
theorem
通过定义不同的 theorem 环境,自己定义定理、定义、引理等的式样。
 
subeqn
提供了 subequations 和 subeqnarray 环境,可以对数学公式中的子式进行编号。得到如 (1a), (1b), (1c) 这样的公式编号。参见 subeqn 的例子。
 
subeqnarray
定义了 subeqnarray 和 subeqnarray* 环境,可对一组公式中的每行进行编号,给出如 (1a), (1b), (1c) 等的编号。参见 subeqnarray 的例子。
 
mathenv
提供了几个很有用的数学命令和环境,可以得到比相应的标准的 LaTeX 命令或环境更好的排版结果。
 
eqnarray
定义了 equationarray 环境,将 LaTeX 标准的 eqnarray 环境和 array 环境结合起来。
 
youngtab
定义两个命令来排版如这样的 Young-Tableaux 式子。
 
yhmath
提供了一系列很大的分界符如: ( ), < > [ ] 等。
 
tmmath
支持用 Adobe Times 和 TM-Math 字族来排版文本和数学公式。
 
vector
提供了一组新的数学命令来排版各种式样的向量。
 
nicefrac
在正文文本中排版分式时,可以用它来得到较好的排版效果。
 
mdwmath
定义了 sqrt* 命令来得到没有上面的横线的根式符号,此外还定义了其它一些数学符号。
 
Bold math symbols
定义了 bm 命令,可用来得到加粗的斜体字体。
 
dstroke
排版如下的 “double stroke”数学符号。
 
ntheorem
扩展了 LaTeX theorem 环境的功能,并解决了设置定理环境的结束标记的问题。
 
easybmat
排版块状矩阵。可以设置相同宽度的列,或登高的行,或两者同时设定。此外,还可以在行或列之间加上各种直线。
 
harpoon
提供了一些命令在文本上方或下方加上带有半个箭头的线段标记。
 
chemsym
由 Mats Dahlgren 设计,目的在于正确地排版化学元素的名称。它提供了 109 条相应于化学元素的命令,其命令名称与元素的化学符号完全一致。
 
xymtex
Shinsaku Fujita 在1993年到1995年期间开发的专门用于绘制化学中有机分子等结构的一组宏,它由一组 LaTeX 宏包组成。
 
ppchtex
是 ConTeXt 中的独立模块,专门用来排版化学符号和公式。
 
抄录和代码打印 
verbatim
重新实现了 LaTeX 的 verbatim 和 verbatim* 环境,并提供了新的环境 comment 和 verbatiminput 来在文挡中加入评论和直接抄录文件。是 LaTeX 标准的工具包 tools 之一。一般的 TeX 软件均包括此宏包。
 
moreverb
应用上面的 verbatim 宏包所提供的命令,对抄录环境进一步加以扩展。主要是增加了与制表符有关的一些功能,行号,将抄录的内容写入一文件以备重复使用等。
 
fancyvrb 与 fvrb-ex
fancyvrb 宏包提供了方便的命令来设计不同式样的抄录环境。如使用不同的字体,颜色,加入行号,边框等。还可根据不同的条件对抄录的文本使用不同的式样。 fvrb-ex 宏包则利用 fanvyvrb 所提供的命令给出了一个 example 环境,允许在列出包含 TeX 命令的文本的同时将该文本排版。
 
sverb
提供 list 等环境,可将抄录环境中的内容写入外部文件中,也可从外部文件中读入。
 
listings
排版 C, C++, Pascal 等源代码,提供语法加亮显示的功能。
 
algorithms
提供排版算法步骤的 algorithmc 和 algorithm 环境,对其中的关键词可采用不同的显示效果。
 
newalg
定义了排版算法步骤的 algorithm 环境。
 
program
排版编程语言的源代码或算法步骤。
 
特殊文本元素
footmisc
提供了许多命令来弥补标准的 LaTeX2e 中 footnote 命令的不足。包括可以用符号替代脚注的数字编号,将脚注放置在边注区,在同一地方使用多个脚注等。
 
footnote
改进了标准的 LaTeX2e 的 footnote 命令,使得可以在 parbox, minipage 和 table 环境中标记的脚注能够被正确地放置在整个页面的下方脚注区中。
 
ftnright
使用这个宏包可以在多栏版式的文档中,将一页上的所有脚注都放置在最右编一栏的底部。而不是放置在各自所在栏的底部。
 
footnpag
自动设置脚注的计数器,使得对每一页上的脚注都可以设置自己的编号。这里的编号不仅仅是数字,也可以是其它符号。该宏包可以很好的配合标准的 LaTeX2e 文档类。
 
savefnmark
可以将 table 或 minipage 环境中的脚注加以标记,并可在后面再次使用。
 
abstract
可以用来方便的设置 abstract 环境,特别是当在双栏版式中排版单栏的简介时。
 
lastpage
将标记 Lastpage 写入 .aux 文件中,允许使用者引用文档的最后一页。比如在页脚可以得到“Page 2 of xxx pages”这样的效果,这里的 “xxx” 就是用 pageref{Lastpage} 得到的文档页码总数。
 
xr
利用此宏包的 externaldocument 命令,可以实现对外部文档的标记的引用。
 
hyperref
扩展了 LaTeX 的所有的交叉引用的命令(包括目录,参考文献等)的功能,使其生成各种驱动如 dvips, pdftex 等可识别的 special 命令,从而得到超文本链接。此外,该宏包还提供了新的命令来支持在文档中加入对外部文档和 Internet 网址的链接。
 
schedule
看名字就知道是排版时间表的宏包。
 
acronym
提供了很多命令帮助你在文档中方便地处理首字母缩略词,并在文档的最后生成一个列表。
 
hypbmsec
扩展了 section 命令,允许在 section 命令中同时给出出现在标签(Bookmarks)和正文中的标题,而这些标题可以有所不同。这是因为出现在 section 命令中的标题不一定符合 PDF 标签的要求,如不能使用 TeX 命令等。
 
hyphenat
可以在文档中取消 TeX 自动断词的功能,也可以在某一单词后再恢复这一功能。
 
units
基于 nicefrac 宏包,提供对计量单位比较美观的排版效果。
 
Slunits
提供对国际标准的计量单位符号的支持。
 
soul
支持对单词加上下划线或其每个字母在一定的宽度内均匀散布。
 
altfont
使用该宏包,可以在一个宏包中使用多种不同的字体,包括 PSNFSS 和 MFNFSS。
 
prelim2e
可以在每页页脚下方标记出本文档的版本信息等。
 
lineno
在每行文本前加上行号,并且可以用 LaTeX 交叉引用来引用它们。
 
moresize
重新定义了 Huge 命令,此外还定义了比其更大的字体。
 
texshade
texshade 使一个 TeX/LaTeX 多序列排序、比较的的软件,它可以用同一种颜色表记出各个序列中相同部分。当然,它的功能不会只有这一点。
 
niceframe
定义了 niceframe 等新的命令,可以将文本等放置在用 dingbat 字体生成的装饰框内。
 
fancybox
提供了 shadowbox, doublebox, ovalbox 和 Ovalbox 四个命令来生成不同形状的盒子。
 
indentfirst
让每一章节开始的段落也缩进。可以和标准的 LaTeX 文档类配合使用。
 
辅助工具包 
pagesel
利用此宏包可以很方便从输出页面中选取一页或多页。
 
count1to
用 page, part, ..., 到 subparagraph 的值设置计数器 count1, ..., count8,而 count9 则用来标记奇数页。通过显示这些计数器的值并将其写入 .dvi 文件中,可以实现对文档的某一部分进行有选择的打印。
 
stdclsdv
对正确识别 LaTeX 的标准类文件中所提供的章节的级别,如有无 chapter 或 section 这一级的命令提供了一个解决办法。
 
showlabels
帮助用户跟踪所有的标记(labels)。每当使用 label 命令或遇到一个自动编号的公式时,就把新的标记的名字放置在所在页的边空中。
 
showkeys
该宏包修改了 label, ref, pageref, cite 和 bibitem 命令,使得这些命令所使用的“内部标记”被显示在边注区或其所得结果的上方,并且尽可能不影响排版的结果,为标准的 LaTeX2e 工具包之一。
 
fileerr
定义了多个文件使得可以很容易地从找不到文件的错误循环中退出,为标准的 LaTeX2e 工具包之一。
 
calc
重新实现了 LaTeX 的命令:setcounter, setlength, addtocounter 和 addtolenght。使得可以在这些命令里使用符号表达式,为标准的 LaTeX2e 工具包之一。
 
changebar
通过在页边空处加上一竖直条来标记 LaTeX 文档中改动过的部分等。
 
alphalph
提供了两个命令 alphalph 和 AlphAlph 可将数字转换为字母。
 
typehtml
使用此宏包可以在 LaTeX 文档中处理 Html 代码。
 
非标准文档式样
seminar
不经意间就做成了令人满意的投影胶片。
 
foiltex
排版幻灯片、胶片,并且可以和 fancybox 配合得到很好的立体效果。
 
pdfslide
排版幻灯片、胶片。配合上 hyperref, 用 pdflatex 编译生成 PDF 并经 ppower4 处理后,可以得到与 powerpoint 相媲美的演示效果。
 
pdfscreen
不用复杂的命令就能设计并得到精美的 PDF 文档。可以让文本显示在不同形状的窗口中,再加上导航按钮,背景。
 
texpower
排版可以在屏幕上演示的投影片。它允许使用 PStricks, XYpic 等pdflatex 所不支持地宏包,但需要用Acrobat Distiller 来得到最后的 PDF 文件。参见用户手册和演示文件(英文 PDF )。
 
KOMA-Script Class
这是一套按照欧洲的排版标准设计的 LaTeX2e 的文档类。与 LaTeX2e 所提供的标准文档类稍有不同。如果在 CJK 中使用中文的章节号,需要使用这套文档类。
 
geom
以标准的 LaTeX 的 article 和 book 类为基础,增加了许多功能。
 

exam
用来排版试题的文档类。
 
draftcopy
在文档的某些页面印上 DRAFT 字样的水印。
 
labels
用来制作地址标签。
 
AMS Book/Paper
美国数学会的书稿和论文的式样文件。
 
paper/journal
是对 article 类的扩充,定义了几个有用的命令来增强对标题和关键词等的处理。
 
a0poster
提供了特大号的字体,可以排版 a0 纸大小的海报。